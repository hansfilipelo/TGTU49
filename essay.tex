\documentclass[a4paper,12pt,fleqn]{article}
\usepackage{fixltx2e}
\usepackage{helvet}
\usepackage[utf8]{inputenc}
\usepackage{graphicx}
\usepackage{sidecap}
\usepackage{fancyhdr}
\usepackage{amssymb,amsmath}
\usepackage[swedish]{babel}
\usepackage[margin=2.5cm]{geometry}
\usepackage{abstract}
\usepackage[parfill]{parskip}
\usepackage{tocloft}
\usepackage{adjustbox}
\usepackage{textcomp}
\usepackage[T1]{fontenc}
\usepackage{listings}
\usepackage{xcolor,colortbl}
\usepackage{hyperref}
\usepackage{mcode}
\usepackage{a4wide}
\usepackage{caption}
\usepackage{helvet}
\usepackage{tocloft}
\usepackage{setspace}
\DeclareUnicodeCharacter{00A0}{ }

% Allow footnotes listing so I don't need to quote other sources
\newcommand{\listfootnotesname}{Referenser}% 'List of Footnotes' title 
\newlistof[chapter]{footnotes}{fnt}{\listfootnotesname}% New 'List of...' for footnotes 
\let\oldfootnote\footnote % Save the old \footnote{...} command 
 \renewcommand\footnote[1]{% Redefine the new footnote to also add 'List of Footnote' entries. 
     \refstepcounter{footnotes}% Add and step a reference to the footnote/counter. 
     \oldfootnote{#1}% Make a regular footnote. 
     \addcontentsline{fnt}{footnotes}{\protect 
 \numberline{\thefootnotes}#1}% Add the 'List of...' entry. 
}

% Set sans serif font to Helvetica
%\setsansfont{Helvetica}
% Set serifed font to Cambria
% \setmainfont{Cambria}

\begin{document}
\onehalfspacing

% Titelsida -----------------------------
\begin{titlepage}
\begin{center}

% Upper part of the page. The '~' is needed because \\
% only works if a paragraph has started.

~\\
~\\
\textsc{\LARGE Link{\"o}pings Universitet}\\[1.5cm]


% Title
~\\
~\\
{ \huge \bfseries Teknikhistoria \\ En redogörelse \\[0.4cm] }

% Author and supervisor
\large
\emph{Av:}\\
Hans-Filip \textsc{Elo}

\vfill

% Bottom of the page
{\large \today}

\end{center}

% Slut på titelsida. ---------------------

% Innehåll ------------------------------
\newpage
\thispagestyle{empty}
\tableofcontents

\newpage
\listoffigures

\end{titlepage}

%header ---------------------------------
\pagestyle{fancy}

\fancyhead{} % clear all header fields
\fancyhead[L]{Hans-Filip Elo \slshape}
\fancyhead[R]{Teknikhistoria - En redogörelse \slshape}

%slut på header ---------------------------------

\section{Inledning}

Människans första kontakt med redskap var i form av skarpt formade stenar som användes för att skära och spetsa föda. Från denna första kontakt har utvecklingen som vi alla vet minst sagt gått långt. Det faktum att denna uppsats skrivs på en högteknologisk skrivmaskin med inbyggd versionshantering och så kallad molnbaserad lagring gör väl denna text till ett så gott bevis som något. 

Tekniken har börjat komma att betyda väldigt mycket för väldigt många. Till en början var redskap och teknik ett sätt att överleva och effektivisera, men idag kan den vara så mycket mer. Tekniken kan idag vara allt ifrån ett ett hjälpmedel, till ett intresse, till ett hinder. 

För undertecknad betalar tekniken min lön och mitt levebröd och den är en starkt utvecklande faktor för min utbildning. Tekniken kan alltså påverka alla våra liv i de mest konstiga riktningar. Tekniken kan också påverka hela vår jord och våra globala förhållanden, dels till det bättre - men också till det sämre. 

\subsection{Begränsningar och förutsättningar}

Denna hemtenta sammanfattar kort teknologins utveckling från forntid till nutid. Om ej annan källa anges har information hämtats från kurslitteraturen som skrivits av Staffan Hansson\footnote{Hansson, Staffan. Den skapande människan. 1:a upplagan. Lund: Studentlitteratur, 2002}. 

% ---------- Forntid
\newpage

\section{Forntiden}

Redskapen under forntiden var väldigt enkla. Det viktigaste redskapet tros varit handkilen, alltså en vass sten med rundare bakstycke som kan hållas i handen. Handkilar (se figur \ref{handkil}\footnote{José-Manuel Benito, Creative Commons 3.0 license, \url{http://creativecommons.org/licenses/by-sa/3.0/}}) har hittats vars ålder daterats till 2,5 miljoner år. Till en början var redskapen stenar man fann med passande form. Allt eftersom upptäckte man däremot att vissa stenar är hårdare än andra - varpå de hårdare stenarterna användes för att skala stenar av mjukare stenarter till rätt form. 

\begin{figure}[ht!]
\centering
\includegraphics[width=70mm]{handkil.png}
\caption{Skiss av handkil}
\label{handkil}
\end{figure}

Handkilen som redskap födde i sin tur andra redskap, framförallt för att tillfångata föda och försvara sig. Med hjälp av handkilen kunde man senare, kring år 40 000 till 10 000 f Kr tillverka mer avancerade redskap såsom kastträ, vassa spjut och harpuner och pilbågar. Från den här tidsperioden har man även funnit grottmålningar.

Forntiden är en väldigt lång tidsperiod, faktiskt större delen av människans existens, men bristen på dokumentation gör att vi får förlita oss på de fynd som gjorts för att uppskata levnadsförhållanden och teknologisk utveckling. Under senare perioder har människan på eget bevåg dokumenterat mer av sitt leverne, vilket gör att kunskapen om dessa tidsperioder är större. 

\subsection{Människan blir jordbrukare}
I takt med människans utbredning minskade tillgången på större rovdjur, människans primära näringskälla. Bristen på mat ledde till att människan sökte alternativa sätt att finna föda. Den tidiga jordbrukarens redskap bestod framförallt av grävkäppen, en krökt pinne som användes för att underlätta sådd. 

Människan började också senare att ha husdjur. Dessa husdjur innefattade hundar, får, getter, hästar och oxar. I takt med att utvecklingen av fler redskap kom till - så som hackan, skäran och även årdret -  kom höstar och oxar att användas i för att bistå med jordbruket. Genom de nya redskapen och djuren kunde jordbruket öka i skala och bättre försörja en familj. 

Jordbruket gjorde också människan till en mycket mer bofast varelse, vilket möjliggjorde mer permanenta bosättningar. 

\subsection{De första civilisationerna}

I och med den mer bofasta levnadsstilen insåg människan fördelen av att samordna sina boenden. En större grupp är tryggare från attacker av djur och andra människogrupper, och invånarna i en större bosättning kan bistå varandra med olika tjänster. De första större civilisationerna skapades kring ekvatorn, runt några av jordens större floder, kring år 4000 till 3000 f Kr. Att civilisationer skapas först kring ekvatorn och floder är ingen tillfällighet, i o m att tillgången på sötvatten är hög och likaså bördigheten i jorden. 

I och med att bosättningar byggdes i allt större skalor skapade människan ett behov för teknisk utveckling. Man började utveckla byggnads- och bevattningstekniker allt mer för att kunna fylla både mat- och logi till sina civilisationers växande befolkningar. Det är i dessa civilisationer man först funnit bevis på att människan arbetat med brons, alltså en legering mellan tenn och koppar. Bronset hade en betydande del framförallt för vapenproduktion, men till viss del också verktyg. 

De tekniska framstegen inom byggnads- och bevattningsteknik gjorde att fler människor kunde få sina levnadsbehov uppfyllda, vilket ledde till att utvecklingen spirade på en rad andra områden. 

I forntidens civilisationer utvecklades det första skriftspråket, i Mesopotamien. Detta skriftspråk, kilskriften, var ett bildligt skriftspråk som använde symboler som beteckning för ord. Egyptens hieroglyfer tros vara influerade av den Mesopotamiska kilskriften. I och med skriftspråket introducerades även tillverkningen av pergament både i Egypten och Mesopotamien. 

% ---------- Antik
\newpage
\section{Antiken}

Efter de forntida civilisationernas storhetstid startade, kring år 700 f Kr, antiken. Under antiken stod tre civilisationer, Grekland, Kina och Rom som de starkast lysande stjärnorna på den kulturella natthimlen. Grekland ses hos många teoretiker och läror som den västerländska kulturens grundare, och Rom som dess spirituella arvtagare. 

Det största teknologiska arvet från denna period är förmågan att behandla järn till stål. Detta var en tidsödande och tung process där järnet värmdes upp och sedan bankades ut minst 200 gånger. Av denna anledning användes stålet i princip enbart till vapen. Förmågan att tillverka stål är däremot något som är viktigt för framtiden och var en stor del i den industriella revolutionen (ca 1750 till 1870). 

Kvarnar för att mala säd uppfanns också under denna period - även om deras spridning i Europa först blev markant under medeltiden (ca 500 till 1500 e Kr). 

\subsection{Grekland}

Grekland - västvärldens kulturella vagga. Det ses som något ädelt. Antikens Grekland är något som de flesta har goda referenser till. Civilisationen utvecklade i princip alla vetenskapliga områden. Alla former av vetenskap och även jämlikhet (dock mellan män) var saker som prioriterades högt. Grekerna grundlade hela vetenskapliga områden så som matematiken, astrologin, filosofin och fysiken. Strävan efter jämlikhet gjorde också att grekerna skapade världens första försök till en demokratisk stat\footnote{Nationalencyklopedin, http://www.ne.se/uppslagsverk/encyklopedi/l\%C3\%A5ng/demokrati/demokratin-under-antiken, hämtad 2014-11-17}. 

\newpage

I Alexandria, i nuvarande Egypten, i antikens Grekland uppfördes år 300 f Kr världens första universitet, Museion , där dåtidens forskare och filosofer verkade. Här verkade bland annat Euklides, geometrins fader. En annan känd matematiker från antiken är Pythagoras, mest känd för ''Pythagoras sats'' ($a^2+b^2=c^2$, se figur \ref{pythagoras}\footnote{Michael Hardy, Creative Commons 3.0 license, \url{http://creativecommons.org/licenses/by-sa/3.0/}}) vilken beskriver förhållandet mellan en rätvinklig triangels sidlängder\footnote{Forsling, Göran, Neymark, Mats. Matematisk analys i en variabel. 2:a upplagan. Linköping: Liber}.

\begin{figure}[ht!]
\centering
\includegraphics[width=70mm]{pythagoras.png}
\caption{Pythagoras sats}
\label{pythagoras}
\end{figure}

Den mest kända vetenskapsmannen från antikens Grekland får nog ändå ses som filosofen och fysiken Archimedes. Archimedes princip beskriver kroppars förmåga att ''flyta'' respektive ''sjunka'' och är en grundsten inom fysiken. 

Det finns en stor del väldigt kända vetenskapsmän och filosofer från antikens Grekland, men dessa får stå som exempel för den accelererade vetenskapsutvecklingen under den här perioden. Dessa vetenskapliga framsteg får också stå som märke för antikens Grekland i denna text. 

\subsection{Romerska Imperiet}

Romarnas kultur baserade sig till stor del på Greklands, man kopierade exempelvis deras religion och gjorde den till sin egen. Romarriket är det största europeiska riket någonsin. Rent teknologiskt var romarna var pionjärer inom bland annat stridsteknik och -taktik samt vägbyggande. Romarna byggde ett massivt vägnät över det gigantiska riket. Handel via fartyg hade sedan tidigare tagit fart, men i och med romarnas vägbyggande kunde nu befolkningen transportera varor med häst och vagn, om än i mindre utsträckning. 

\subsection{Kinesiska riket}

Kinesiska riket, Mittens rike, var vid den här tiden det teknologiska meckat. Kineserna var i många tekniska tillämpningar århundraden före européerna. Kineserna var framförallt bättre på att gruvverksamhet och metallraffinaderi. Detta i sin tur ledde till många andra teknologiska framsteg som var beroende av järn och stål. 

Kineserna utvann exempelvis salt ur jorden genom att borra med stora stötborrar. Dessa borrar var tillverkade i stål och metallraffineringen var en förutsättning för denna industri. 

Kineserna som också de hade ett stort rike vid den här tiden, hade precis som romarna ett välutvecklat vägnät. Det välutvecklade vägnätet ledde också till att man utvecklade brobyggandet markant. Transporter var helt enkelt något kineserna var bra på, då de också byggde för tiden gigantiska skepp för transport och handel. 

Eftersom civilisationen var stor och välbefolkad behövde man transportera vatten från källa till bebyggelser. Framstegen inom brobyggande i kombination med behovet av vatten ledde till byggande av akvedukter. Vattentrycket i antikens vattentappar byggde på kontinuerligt sluttning hela vägen från källa till mål. Då lutningen vare sig fick vara för stor eller för låg spelade akvedukter en stor roll här. 

Kinesernas förbättrade även seldon för hästar. Där tidigare seldon (före 200 f Kr) ströp hästen vid tung last var kinesernas nya seldon mer ergonomiskt. Detta medförde att hästarna kunde dra tyngre släp en längre sträcka. 

% ---------- Medeltiden 
\newpage
\section{Medeltiden}

Medeltiden i Europa inleddes kring år 500 e Kr. I historieböckerna kännetecknas denna tid ofta av kyrkans förtryck av vetenskapen, Digerdöden och feodalsystemets förtryck av individen. Då medeltiden varade fram till cirka år 1500 rörde det sig ändå om 1000 år av mänsklighet, vilket som vanligt ändå mynnade ut i teknologiska framsteg. Ett exempel är att Masugnar (ugn för att utvinna järn ur järnmalm) började användas i Europa under denna tid.

\subsection{Jordbruket}

Under medeltiden utvecklades jordbruket en hel de. I Europa började man använda hjulplogen, harven, bättre seldon samt hästskon. Dessa innovationer inom jordbruket ledde till att större skördar kunde göras och därmed var behovet att mala även det större. Av denna anledning både ökade antalet kvarnar markant, och de blev också mer avancerade. Både värderkvarnen och vattenkvarnen var något som, i Europa, först började användas i större utsträckning under denna tid. 

En vattenkvarn vid ett livligt vattendrag möjliggjorde en kontinuitet i drift som vare sig väderkvarnen eller manuell drift kunde mäta sig med. Kontinuiteten möjliggjorde större produktioner av mjöl. 

\subsection{Boktryckarkonsten}

Förmågan att läsa och uttrycka sig i skrift är en viktig grundsten i det moderna informationssamhället. Boktryckarkonsten var en central del i att kunna distribuera böcker till fler ur befolkningen. Den ledde till en kraftig minskning i kostnad per bok. Andelen analfabeter under medeltiden var stort. Man kan argumentera för att boktryckarkonsten var medeltidens största bidrag till den moderna världen, även om den kanske inte gjorde ett så stort avtryck under sin tid. 

\subsection{Tideräkning med mekaniska ur}

Tideräkning har av förklarliga skäl utförts med hjälp av solen sedan tidernas begynnelse. Precisionen har i takt med tiden blivit bättre och bättre. Först kom soluren, vilka stod sig länge som de mest precisa klockorna. Under medeltiden blev mekaniska urverk allt vanligare. På 1500-talet kom de första fickuren, dessa hade dock en väldigt dålig precision och kunde ha en tidsdifferens på timmar över ett dygn. Under 1600-talet blev däremot fickuren mer precisa och de bästa fickuren hade bara ett fel på några minuter över 24 timmar. 

Revolutionen mekaniska ur ledde till att man i samhället, framförallt senare under industriella revolutionen, började räkna arbetstimmar och tid på ett allt noggrannare sätt. 

% ---------- Industriella revolutionen 
\newpage
\section{Industriella revolutionen}

Kring mitten på 1700-talet började man upptäcka att om man ökade produktionen så kunde man minska kostnaderna, och därmed göra mer pengar. Vinstintresset i kombination med kyrkans minskade status i samhället och den mer utvecklade vetenskapen ledde världen in i en ny era. 

Redan på 1500-talet såg man jordbruket bli mer samverkat i Storbritannien. Istället för att odla själv på mindre åkrar utökade man åkrarna för att mer effektivt kunna nyttja marken. Denna tanke fördes sedan in i 1700-talets industrialisering. Större skala leder till större volym, vilket i sin tur leder till lägre kostnad per enhet. 

Det var under denna tid som Europa etablerade sig som den stora kulturella och teknologiska drivkraften. Den industriella revolutionen anses ofta gå över i den moderna tiden någonstans mellan år 1800 och 1900. 

\subsection{Verktygen förbättras}

Ett viktigt led i att effektivisera produktion var, och är, att skapa bättre hjälpmedel för framställning. Viktiga uppfinningar som bidrog till att dryga ut arbetarnas förmåga att producera varor var bland andra spinnmaskinen ''Spinning Jenny'' och ångmaskinen. ångmaskinen utvecklades under denna period från från simplare pumpar till mer avancerade och specificerade maskiner. Stor del av denna utveckling tillskrevs skotten James Watt och svensken Gustav de Laval . 

\subsection{Metalltillverkning}

Förmågan att utvinna och förädla metaller har alltid varit centralt i människans teknologiska utveckling. Det är en viktig del i krigsföring, men också i att bygga hållbara och effektiva verktyg. I 1700- och 1800-talets Storbritannien ledde den ökade industrialiseringen till en brist på träkol. Detta ledde till att man började utvinna och använda mer Stenkol. Stenkolet kan vid eldning ge en högre effekt än träkol varpå järnutvinning förenklades. Stenkolet användes bland annat för att smälta järnmalm för att utvinna järn, och också för att producera stål från denna järnmalm. 

ångmaskinen är ett exempel på en innovation under industriella revolutionen som var helt beroende av metallutvinningen. 

\subsection{Centralisering - och decentralisering}

Under denna tid skedde en stor befolkningsökning till följd av den ökade produktionen. Fler hade sina essentiella behov fyllda helt enkelt. Befolkningsökningen och den förändrade produktionen av föda och varor ledde till en centralisering i Europa. Allt fler människor flyttade till städer, och hela städer kom senare att skapas kring fabriker. 

Samtidigt som centraliseringen i Europa pågick var de starka nationernas vilja att utveckla kolonier runtom världen allt större. Industrialiseringen gav en allt större efterfrågan på varor, och gav kolonierna en allt större betydelse för att produktion av dyrare varor och arbetskraft i form av slavar. 

De europeiska nationerna koloniserade övriga delar av världen och höll sina kolonier i ett hårt grepp. Större delar av Afrika, Sydostasien och Amerika var alla kolonier till europeiska stater under denna tid. 


% ---------- "Modern tid"
\newpage
\section{Modern tid}

I den moderna tiden har en mängd tekniker och uppfinningar sett dagens ljus. Allt från flygplan och bilar till kärnkraft och solceller till datorer och internet. Att gå igenom dem alla skulle vara ett väldigt stort antagande, allt för stort för omfattningen av denna uppsats, men jag kommer här att nämna en del som kommit att definiera vår den moderna tiden.  

Den moderna tidens starkaste redskap har kommit att bli elektriciteten. Allt vi gör är beroende av, eller använder på något sätt, elektricitet. Ett exempel på detta är att framställning av stål, en tidigare så svår och viktig del av människans teknologi, nu görs i klart större mängder med hjälp av elektricitet och väldigt lite arbetskraft.

Den moderna tidens tekniker och vägar att nyttja våra naturtillgångar har lett till helt nya användningar. Exemplet med den enorma stålproduktionen har lett till helt nya användningsområden för materialet. Man använder stål, aluminium och andra metaller till väldigt mycket, exempelvis i bilar, broar, båtar, flygplan, förvaringsburkar m m. 

1900-talets teknologiska utveckling har till stor del influerats av två världskrig, samt av det ''Kalla kriget'' mellan två supermakter. Att säga att dessa faktorer stått för \textit{all} teknologisk utveckling vore fel, men man kan absolut inte förbise dess betydelse. 

\subsection{Världskrigen}

Under första och andra världskrigets kritiska lägen tvingades människan vidareutveckla vetenskapliga och teknologiska principer till färdiga krigsmateriel. Man såg här utvecklingen av en mängd tekniker till brukbart skick, däribland stridsvagnar, maskingevär, radar, jetmotorer, raketer, datorer, helikoptrar och kärnvapen. Många av dessa tekniker kom att definiera 1900-talets maktstruktur. 

\subsection{Kalla kriget}

Kalla kriget uppstod i spänningarna kring efterkrigstiden från 1945 och framåt. De allierade delades upp i två läger, öst- och västblocket. USA ansågs vara västblockets ledare och Sovjetunionen östblockets. Dessa två block delade framförallt Europa men också delar av andra världen i två läger, NATO och Warszawapakten, se figur \ref{coldWar}\footnote{San Jose, Creative Commons 3.0 license, \url{http://creativecommons.org/licenses/by-sa/3.0/}} där blå är NATO och röd är Warszawapakten. 

\begin{figure}[ht!]
\centering
\includegraphics[width=70mm]{cold_war.png}
\caption{NATO vs Warszawapakten}
\label{coldWar}
\end{figure}

Under Kalla krigets tid utvecklats raket- och rymdteknologi, kärnvapen, ubåtar, flygplan och datorer kraftigt. Framförallt USA var den teknologiskt drivande nationen, även om USA och Sovjetunionen konstant försökte trumfa varandra med tekniska framsteg. 

Den teknologiska och politiska kampen mellan dessa två supermakter utspelades på flera teknologiska plan, dock störstadels inom militära tillämpningar. Under slutet av 1950 fram till början av 1970-talet utspelade sig den så kallade ''Rymdkapplöpningen'', som påbörjades med världens första artificiella satellit Sputnik år 1957 och avslutades med månlandningarna år 1969-1973. Viktigt om rymdkapplöpningen är att den gav en möjlighet att utveckla och påvisa sin raketteknologi för världen, vilken var essentiell i utvecklingen av s k Intercontinental Ballistic Missiles (ICBM)\footnote{Intercontinental Ballistic Missile, Wikipedia \url{http://en.wikipedia.org/wiki/Intercontinental_ballistic_missile}, hämtad 2014-12-10}. 

En kärnvapenbestyckad ICBM kan utplåna hela städer, och det finns egentligen inget skydd mot denna. Ett krig mellan USA och Sovjetunionen hade på grund av detta vapen skapat stor förödelse för mänskligheten. 

\subsection{Tiden efter Sovjetunionens fall}

Efter Sovjetunionens fall (1989-1991) stabiliserade det världspolitiska klimatet något. Mindre pengar satsades på militär verksamhet och sociala problem samt naturproblem blev ett större fokus i många civiliserade länder. I denna period kom också ett gigantiskt datornätverk, kallat ''Internet'', att slå igenom. I och med internet startade en stor informations- och kommunikationsrevolution som förenklade för människor att kommunicera med varandra och publicera material till en större massa. 

Denna revolution har lett till stora teknologiska framsteg inom framförallt datorer och datormjukvara. Revolutionen har också startat en större mängd mångmiljonföretag inom teknikbranschen med stora inkomstkällor som ytterligare driver på den teknologiska utvecklingen\footnote{Fortune500, \url{http://fortune.com/fortune500/}, hämtad 2014-12-09}. 

% -------------- Slutsats?
\newpage
\section{Framtiden och det teknologiska arvet}

Att den teknologiska utvecklingen gått hand i hand med den sociala, ekonomiska och vetenskapliga utvecklingen är inget att tveka på. Ibland kommer föder den ekonomiska eller sociala utvecklingen ett behov hos tekniken eller vetenskapen, ibland förändrar tekniken eller den vetenskapliga utvecklingen spelfältet för det sociala och ekonomiska landskapet. 

Gemensamt är att teknisk utveckling som sker i öppenhet, eller publika domänen, är det som gett mest mervärde för mänskligheten. Utveckling av säkerhetskritiska system går allt mer mot att ske i en publik domän\footnote{OpenSSL, \url{https://www.openssl.org/}, hämtad 2014-12-08}. Spionage och brist på öppenhet har förr, framförallt under Kalla Kriget, varit en stor faktor till spänningar inom världspolitiken. Fortsatt brist på öppenhet och det allt mer utbredda cyberespionaget kan vara ett stort hot mot det moderna samhället. 

\subsection{Miljö}

Den moderna tidens förmåga förmåga att förädla naturtillgångar i stora mängder har lett till att människan i välutvecklade länder lever i ett stort överflöd - och även skapat en vana att leva i ett stort överflöd\footnote{Miljösituationen i världen, Svenska FN-förbundet: \url{http://www.fn.se/PageFiles/2675/UNEPrapporterarommiljosituationenivarlden.pdf}, hämtad 2014-12-07}. 

Människans vana att leva i överflöd i kombination med människans skenande population, som möjliggjorts av viktiga medicinska framsteg\footnote{5 medicinska upptäckter som förändrat världen, Allt om Vetenskap: \url{http://www.alltomvetenskap.se/nyheter/5-medicinska-upptackter-som-forandrat-varlden}, hämtad 2014-12-07}, leder till att jordens resurs krymper i allt större takt och det blir svårare att utvinna naturtillgångar ur jorden. Vi har också problem med klimatförändringar\footnote{Union of Concerned Scientists, \url{http://www.ucsusa.org/global_warming}, hämtad 2014-12-07}. Dessa miljöhot kan argumenteras vara det moderna samhällets största problem och kan på lång sikt förstöra möjligheter för mänskligheten att frodas. 

Det kan onekligen sägas vara en intressant framtid vi står inför.

% --------- Källor

\newpage
\addcontentsline{toc}{section}{Referenser}

\listoffootnotes

\end{document}




