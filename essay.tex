\documentclass[a4paper,12pt,fleqn]{article}
\usepackage{fixltx2e}
\usepackage[utf8]{inputenc}
\usepackage{graphicx}
\usepackage{sidecap}
\usepackage{fancyhdr}
\usepackage{amssymb,amsmath}
\usepackage[swedish]{babel}
\usepackage[margin=1.5in]{geometry}
\usepackage{abstract}
\usepackage[parfill]{parskip}
\usepackage{tocloft}
\usepackage{adjustbox}
\usepackage{textcomp}
\usepackage[T1]{fontenc}
\usepackage{listings}
\usepackage{xcolor,colortbl}
\usepackage{hyperref}
\usepackage{mcode}
\usepackage{a4wide}
\usepackage{caption}

\begin{document}

% Titelsida -----------------------------
\begin{titlepage}
\begin{center}

% Upper part of the page. The '~' is needed because \\
% only works if a paragraph has started.

~\\
~\\
\textsc{\LARGE Link{\"o}pings Universitet}\\[1.5cm]


% Title
~\\
~\\
{ \huge \bfseries Teknikhistoria \\ En redogörelse \\[0.4cm] }

% Author and supervisor
\large
\emph{Av:}\\
Hans-Filip \textsc{Elo}

\vfill

% Bottom of the page
{\large \today}

\end{center}

% Slut på titelsida. ---------------------

% Innehåll ------------------------------
\newpage
\thispagestyle{empty}
\tableofcontents
~\\

\begin{center}
\line(1,0){400}
\end{center}

\listoffigures
~\\

\begin{center}
\line(1,0){400}
\end{center}

\listoftables

\end{titlepage}

%header ---------------------------------
\pagestyle{fancy}

\fancyhead{} % clear all header fields
\fancyhead[L]{TGTU49 - Teknikhistoria\slshape}
\fancyhead[R]{\today \slshape}

\fancyfoot{} % clear all footer fields
\fancyfoot[L,R]{\thepage}
\fancyfoot[L]{Teknikhistoria - En redogörelse}

%slut på header ---------------------------------

\section{Inledning}

Människans första kontakt med redskap var i form av skarpt formade stenar som användes för att skära och spetsa föda. Från denna första kontakt har utvecklingen som vi alla vet minst sagt gått långt. Det faktum att denna uppsats skrivs på en högteknologisk skrivmaskin med inbyggd versionshantering och så kallad molnbaserad lagring gör väl denna text till ett så gott bevis som något. 

Tekniken har börjat komma att betyda väldigt mycket för väldigt många. Till en början var redskap och teknik ett sätt att överleva och effektivisera, men idag kan den vara så mycket mer. Tekniken kan idag vara allt ifrån ett ett hjälpmedel, till ett intresse, till ett hinder. 

För undertecknad betalar tekniken min lön och mitt levebröd och den är en starkt utvecklande faktor för min utbildning. Tekniken kan alltså påverka alla våra liv i de mest konstiga riktningar. 

% ---------- Forntid
\newpage

\section{Forntiden}

Redskapen under forntiden var väldigt enkla

% ---------- Antik

% ---------- Slutsats
\newpage

\section{Slutsats}

Att den teknologiska utvecklingen gått hand i hand med den sociala, ekonomiska och vetenskapliga utvecklingen är inget att tveka på. Ibland kommer föder den ekonomiska eller sociala utvecklingen ett behov hos tekniken eller vetenskapen, ibland förändrar tekniken eller den vetenskapliga utvecklingen spelfältet för det sociala och ekonomiska landskapet. 

Gemensamt är att teknisk utveckling som sker i öppetnhet, eller publika domänen, är det som gett mest mervärde för mänskligheten. En förhoppning från min sida mot framtiden är att allt mer utveckling, framförallt av säkerhetskritiska system, sker i en publik domän. Spionage och brist på öppenhet har förr, farmförallt under Kalla Kriget, varit en stor faktor till spänningar inom världspolitiken. Bristen på öppenhet och det allt mer utbredda cyberespionaget kan vara det största framtida teknologiska hotet mot mänskligheten. 

\end{document}




